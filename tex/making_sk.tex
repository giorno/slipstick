% common packages, fonts, styles and commands
% unlocalised content
\documentclass[a4paper,9pt]{article}
\usepackage{fontspec}
\usepackage[a4paper,lmargin=1.7cm,rmargin=1.7cm,bottom=2cm,top=2cm]{geometry}
\usepackage{multicol}
\usepackage{titling}
\usepackage{fancyhdr}
\usepackage{paralist}
\usepackage{gensymb}
\setmainfont{Roboto Slab Light}

% model name
\newcommand{\modelname}{SR-M1A1}

% section without title
\newcommand{\nosection}[1]{%
  \refstepcounter{section}%
  \addcontentsline{toc}{section}{\protect\numberline{\thesection}#1}%
  \markright{#1}}

% title formatting
\newfontfamily\headingfont[]{Oswald}
\newfontfamily\constfont[]{Roboto}
\newcommand{\constt}[1]{{\constfont\selectfont{\textit{#1}}}}
\renewcommand{\maketitlehooka}{\headingfont}
\setlength{\droptitle}{-5em}

% paragraph spacing
\setlength{\parindent}{0em}
\setlength{\parskip}{0.5em}


\usepackage{polyglossia}
\setdefaultlanguage{slovak}
\newcommand{\makefulltitle}{Ako vyrobiť Logaritmické pravítko }
% custom page number
\fancypagestyle{plain}{% title page
  \renewcommand{\headrulewidth}{0pt}%
  \fancyhf{}%
  \fancyfoot[L]{\headingfont{http://wheel.creat.io/sr}}%
  \fancyfoot[R]{\headingfont{\makefulltitle \textbf{\thepage}}}%
  \renewcommand{\headrulewidth}{0pt}
}
\pagestyle{fancy}% other pages
\fancyhf{}
\lfoot{\headingfont{http://wheel.creat.io/sr}}
\rfoot{\headingfont{\makefulltitle \textbf{\thepage}}}
\renewcommand{\headrulewidth}{0pt}


% common part of how-to perex
\newcommand{\makeperex}{Logaritmické pravítko je analógový počítač - zariadenie, ktoré dokáže vykonať všetky základné a niektoré pokročilé matematické operácie. Nie je také presné ako kalkulačka, ale nepotrebuje elektrinu. Ešte pred päťdesiatimi rokmi bolo bežne používané, neskôr ho nahradili kalkulačky. Logaritmické pravítko používali aj inžinieri a astronauti počas vesmírnych misií na výpočet letových parametrov. }
% custom page number
\fancypagestyle{plain}{% title page
  \renewcommand{\headrulewidth}{0pt}%
  \fancyhf{}%
  \fancyfoot[L]{\headingfont{http://wheel.creat.io/sr}}%
  \fancyfoot[R]{\headingfont{\makefulltitle \textbf{\thepage}}}%
  \renewcommand{\headrulewidth}{0pt}
}
\pagestyle{fancy}% other pages
\fancyhf{}
\lfoot{\headingfont{http://wheel.creat.io/sr}}
\rfoot{\headingfont{\makefulltitle \textbf{\thepage}}}
\renewcommand{\headrulewidth}{0pt}





\title{\fontsize{60}{60}\selectfont LOGARITMICKÉ PRAVÍTKO}
\preauthor{}\postauthor{}\author{}
\predate{}\postdate{}\date{}
\begin{document}

  \begin{center}
    \headingfont\fontsize{32}{32}\selectfont AKO VYROBIŤ
  \end{center}

  {\let\newpage\relax\maketitle}% no new line before title
  \nosection{}
  \large\textbf{\makeperex Tento návod popisuje ako vyrobiť papierové Logaritmické pravítko \modelname.}

  \begin{multicols*}{3}
  \footnotesize Tento zošit obsahuje hárky hrubého papiera, z ktorých vyrežeš, vystrihneš, ohneš a zlepíš \textbf{diely} Logaritmického pravítka:
    \begin{inparaenum}[\itshape a\upshape)]
      \item dva rovnaké hárky označené \textbf{PEVNÁ ČASŤ + BEŽEC}, obsahujúce telá Pevnej časti a Bežca;
      \item dva rovnaké hárky označené \textbf{POSUVNÁ ČASŤ}, obsahujúce telá Posuvnej časti; a
      \item jeden hárok pauzovacieho papiera, označený \textbf{PRIESVITNÉ ČASTI}, ktorý obsahuje priesvitné okienka.
    \end{inparaenum}

  Tento dokument opisuje výrobu jedného Logaritmického pravítka.a

  \normalsize{Predtým ako začneš, budeš potrebovať:
    \begin{enumerate}
      \setlength{\parskip}{0pt}
      \setlength{\parsep}{0pt}
      \item nožnice
      \item ostrý hobby nôž (prípadne nožnice) na vyrezávanie dielov
      \item pravítko, najlepšie kovové, na rezanie rovných hrán
      \item PVA (polyvinylacetát) lepidlo
      \item knihársku kosť
      \item tenkú obojstrannú lepiacu pásku, 4 až 10 mm širokú (môže byť nahradená lepidlom za cenu straty presnosti koncového produktu)
    \end{enumerate}

  Pre ochranu pred zraneniam a nechcenými výsledkami:
      buď opatrný/opatrná s ostrými nástrojmi,
      použi nôž a pravítko pre dokonale rovné rezy,
      buď trpezlivý/trpezlivá a dbaj na detaily,
      umy a vysuš si ruky predtým ako začneš.

  \makesectiontitle{1 Začni s Pevnou časťou}

\footnotesize Pevná časť je vytlačená na hárku \textbf{PEVNÁ ČASŤ + BEŽEC}, je to tá väčšia časť z dvoch.\normalsize

Vyrež Pevnú časť pozdĺž vonkajších čiar (obdĺžnik) vytlačených na lícnej strane hárku a použi nôž a pravítko.

Tlakom knihárskej kosti pretlač hrany pre ohnutie medzi krátkymi čiarami naznačenými na lícnej strane. Jednu takú hranu medzi stupnicami \textbf{ST} a \textbf{cm}, druhú medzi \textbf{inches} a \textbf{L}.

Prehni podľa pretlačených hrán.

  \makesectiontitle{2 Pohyblivá časť}

\footnotesize Pohyblivá časť je vtlačená na hárku \textbf{POHYBLIVÁ ČASŤ}.\normalsize

Vyrež Pohyblivú časť pozdĺž plných čiar na rube hárku, z jedného konca na druhý.

Tlakom knihárskej kosti vyznač hranu pozdĺž čiarkovanej čiary.

Prehni na tejto čiare lícovou stranou nahor.

  \makesectiontitle{3 Okienko Pevnej časti}

\footnotesize Okienko Pevnej časti je vytlačené na hárku označenom \textbf{PRIESVITNÉ ČASTI}hárku označenom \textbf{PRIESVITNÉ ČASTI}. \normalsize

Vyrež okienko Pevnej časti nožom a pravítkom pozdĺž plných čiar.

  \makesectiontitle{4 Dokonči Pevnú časť}

Na vnútornej strane Pevnej časti sa nachádzajú dve šrafované línie. Nalep na ne obojstrannú lepiacu pásku.

Vlož Pohyblivú časť (stranou so stupnicami \textbf{B}, \textbf{CI} a \textbf{C} nahor) a okienko Pevnej časti dnu (okienko navrchu). Napoly zatvor Pevnú časť, aby všetky hrany medzi troma časťami zarovnali.

Odstráň kryciu fóliu z obojstrannej lepiacej pásky vo vnútri Pevnej časti.

Opatrne pritlač hornú polovicu Pevnej časti na priesvitné okienko.

Posuň Pohyblivú časť tak, aby sa index 1 na stupnici B nachádzal na indexe 1 na stupnici A Pevnej časti.

Začni priláčať spodnú polovicu Pevnej časti na priesvitné okienko zľava doprava tak, aby sa index 1 na stupnici C nachádzal na indexe 1 stupnice D na Pevnej časti.

Pokračuj kým nie je Pevná časť dokončená.

   \makesectiontitle{5 Začni s Bežcom}

Vyrež telo Bežca z hárku označeného \textbf{PEVNÁ ČASŤ + BEŽEC} pomocou noža a pravítka pozdĺž vonkajších čiar.

Vystrihni nožnicami oblúkové okienko.

Použi knihársku kosť na pretlačenie hrán, ktoré sú naznačené na lícnej strane krátkymi čiarami.

Prehni na pretlačených hranách.

  \makesectiontitle{6 Okienko Bežca}

\footnotesize Polopriehľadné okienko sa nachádza na hárku pauzovacieho papiera, označenom \textbf{PRIESVITNÉ ČASTI}.\footnote{Hárok obsahuje dve okienka Pevnej časti a dve okienka Bežca. Potrebuješ jedno z každého pre výrobu jedného pravítka.}\normalsize

Pretože niektoré obrysové čiary by ovplyvňovali funkciu Bežca, sú len naznačené v rohoch okienka číselným poradím rezov ( 1, 2, 3, 4 ).

Najskôr urob nožom a pravítkom rovný rez medzi čiarkami označenými 1, potom zopakuj medzi čiarkami 2, atď., až kým nevyrežeš celé obdĺžnikové okienko.

Nalep krátke kúsky obojstrannej lepiacej pásky na rub Bežca pri oblúkovej a rovnej hrane (šrafované oblasti). Odstráň kryciu fóliu.

Opatrne umiestni okienko Bežca na  lepiacu pásku tak, aby ľavá a pravá hrana súladili s hranami Bežca. Pritlač diely k sebe.

  \makesectiontitle{7 Dokonči Bežec}

Bežec sa posúva po vonkajšom povrchu Pevnej časti. Navleč ho na dokončenú Pevnú časť, logom navrch.

Nalep obojstrannú lepiacu pásku na šrafovanú plochu na zadnej strane Bežca, odstráň kryciu fóliu a spoj so zadnou plochou Bežca tak, aby zostala dostatočná medzera medzi Pevnou časťou a Bežcom, aby sa mohli vzájomne pohybovať. 

  \makesectiontitle{8 Skladací návod}

V tomto zošite je ďalší návod, nazvaný \textbf{Ako používať Logaritmické pravítko}, na ktorom sú štyri čiarkované sivé čiary.\footnote{Zošit obsahuje dve kópie spomínaného návodu, jednu pre každé pravítko.} Odstrihni alebo odrež ľavý a pravý okraj návodu podľa vonkajších čiarkovaných čiar. Prehni pozdĺž zostávajúcich dvoch čiar do tvaru leporela. Takto má návod ideálny tvar a rozmer na to, aby sa dal vsunúť do vnútra Posuvnej časti pravítka. Môžeš ho tam dokonca za jednu stranu prilepiť.

  \makesectiontitle{9 Nauč sa používať Logaritmické pravítko}

Prečítaj si inštrukcie v návode \textbf{Ako používať Logaritmické pravítko} a nauč sa používať svoj analógový počítač.

  }
  \end{multicols*}
  
\end{document}
