% common packages, fonts, styles and commands
% unlocalised content
\documentclass[a4paper,9pt]{article}
\usepackage{fontspec}
\usepackage[a4paper,lmargin=2cm,rmargin=2cm,bottom=2cm,top=2cm]{geometry}
\usepackage{multicol}
\usepackage{titling}
\usepackage{fancyhdr}
\usepackage{paralist}
\usepackage{gensymb}

\setmainfont{Roboto Slab Light}

% section without title
\newcommand{\nosection}[1]{%
  \refstepcounter{section}%
  \addcontentsline{toc}{section}{\protect\numberline{\thesection}#1}%
  \markright{#1}}

% title formatting
\newfontfamily\headingfont[]{Oswald}
\newfontfamily\constfont[]{Roboto}
\newcommand{\constt}[1]{{\constfont\selectfont{\textit{#1}}}}
\renewcommand{\maketitlehooka}{\headingfont}
\setlength{\droptitle}{-5em}

% paragraph spacing
\setlength{\parindent}{0em}
\setlength{\parskip}{0.5em}


\usepackage{polyglossia}
\setdefaultlanguage{slovak}
\newcommand{\makefulltitle}{Ako vyrobiť Logaritmické pravítko }
% custom page number
\fancypagestyle{plain}{% title page
  \renewcommand{\headrulewidth}{0pt}%
  \fancyhf{}%
  \fancyfoot[L]{\headingfont{http://wheel.creat.io/sr}}%
  \fancyfoot[R]{\headingfont{\makefulltitle \textbf{\thepage}}}%
  \renewcommand{\headrulewidth}{0pt}
}
\pagestyle{fancy}% other pages
\fancyhf{}
\lfoot{\headingfont{http://wheel.creat.io/sr}}
\rfoot{\headingfont{\makefulltitle \textbf{\thepage}}}
\renewcommand{\headrulewidth}{0pt}


% common part of how-to perex
\newcommand{\makeperex}{Logaritmické pravítko je analógový počítač - zariadenie, ktoré dokáže počítať väčšinu základných a niektoré pokročilé matematické operácie. Nie je síce také presné a nepoužíva sa tak jednoducho ako kalkulačka, ale elegantne demonštruje matematické zákony. Je to historický artefakt, ktorý bol ešte pred päťdesiatimi rokmi bežne používaný v technických disciplínach, inžinieri a astronauti ním počas vesmírnych misií dokonca počítali trajektórie a letové parametre. }
% custom page number
\fancypagestyle{plain}{% title page
  \renewcommand{\headrulewidth}{0pt}%
  \fancyhf{}%
  \fancyfoot[L]{\headingfont{http://wheel.creat.io/sr}}%
  \fancyfoot[R]{\headingfont{\makefulltitle \textbf{\thepage}}}%
  \renewcommand{\headrulewidth}{0pt}
}
\pagestyle{fancy}% other pages
\fancyhf{}
\lfoot{\headingfont{http://wheel.creat.io/sr}}
\rfoot{\headingfont{\makefulltitle \textbf{\thepage}}}
\renewcommand{\headrulewidth}{0pt}





\title{\fontsize{60}{60}\selectfont LOGARITMICKÉ PRAVÍTKO}
\preauthor{}\postauthor{}\author{}
\predate{}\postdate{}\date{}
\begin{document}

  \begin{center}
    \headingfont\fontsize{20}{20}\selectfont AKO VYROBIŤ
  \end{center}

  {\let\newpage\relax\maketitle}% no new line before title
  \nosection{}
  \large\textbf{\makeperex Tento návod popisuje ako vyrobiť papierové Logaritmické pravítko \modelname.}

  \begin{multicols*}{3}
  \normalsize{Predtým ako začneš, budeš potrebovať:
    \begin{enumerate}
      \setlength{\parskip}{0pt}
      \setlength{\parsep}{0pt}
      \item nožnice
      \item ostrý hobby nôž (prípadne nožnice) na vyrezávanie dielov a pravítko, najlepšie kovové, na rezanie rovných hrán
      \item PVA (polyvinylacetát) lepidlo
      \item úzky štetec na nanášanie lepidla
      \item jeden A4 hárok 300 g papiera na posilnenie zadnej steny Pevnej časti pravítka
      \item aspoň štyri A4 hárky starého papiera alebo novín
      \item ťažké knihy pre zaťaženie lepených povrchov
    \end{enumerate}

  \textbf{Pravidlá hry} pre predchádzanie zraneniam a nechceným výsledkom:
    \begin{inparaenum}[\itshape a\upshape)]
      \item buď opatrný/opatrná s nožom a nožnicami;
      \item použi nôž a pravítko pre dokonalé rovné rezy;
      \item buď trpezlivý/trpezlivá a dbaj na detaily;
      \item umy a vysuš si ruky predtým ako začneš;
      \item vždy odstráň pretečené lepidlo;
      \item nerieď lepidlo vodou; a
      \item stlač lepené časti k sebe okamžite po nanesení lepidla.
    \end{inparaenum}

  Tento zošit obsahuje niekoľko hárkov hrubého papiera, z ktorých vyrežeš, vystrihneš, ohneš a zlepíš \textbf{diely} Logaritmického pravítka:
    \begin{inparaenum}[\itshape a\upshape)]
      \item dva rovnaké hárky označené \textbf{PEVNÁ ČASŤ + BEŽEC}, obsahujúce telá Pevnej časti a Bežca;
      \item dva rovnaké hárky označené \textbf{POSUVNÁ ČASŤ}, obsahujúce telá Posuvnej časti; a
      \item jeden hárok pauzovacieho papiera, označený \textbf{PRIESVITNÉ ČASTI}, ktorý obsahuje priesvitné okienka.
    \end{inparaenum}

  Tento dokument opisuje výrobu jedného Logaritmického pravítka.

  \textbf{Krok 1 Začni s Pevnou časťou} Vyrež Pevnú časť z hárku označeného \textbf{PEVNÁ ČASŤ + BEŽEC}, je to tá väčšia časť z dvoch. Rež pozdĺž vonkajších čiar (veľký obdĺžnik) vytlačených na lícnej strane hárku a použi nôž a pravítko (alebo nožnice). Otoč vyrezaný diel rubom nahor a miernym tlakom noža nadrež pozdĺž čiarkovaných čiar. Prehni na nadrezaných čiarach.

  \textbf{Krok 2 Zosilni Pevnú časť} Z hrubého papiera vyrež pruh s rozmermi 62x277 mm. Nanes tenkú vrstvu lepidla na vyšrafovanú plochu na zadnej časti a opatrne naň pritlač vyrezaný pruh hrubého papiera. Uhlaď papier prstami. Vlož zlepený diel medzi dva hárky starého papiera a zaťaž knihami. Nechaj jeden deň schnúť.

  \textbf{Krok 3 Pohyblivá časť} Pohyblivá časť pravítka je vytlačená na hárku označenom \textbf{POHYBLIVÁ ČASŤ}. Rež pozdĺž plných čiar na rube hárku, z jedného konca na druhý. Nadrež pozdĺž čiarkovanej čiary. Prehni na tejto čiare lícovou stranou nahor. Pohyblivá časť je hotová. Jedna z troch!

  \textbf{Krok 4 Začni s Bežcom} Vyrež telo Bežca z hárku označeného \textbf{PEVNÁ ČASŤ + BEŽEC} pomocou noža a pravítka pozdĺž vonkajších čiar. Vystrihni nožnicami oblúkové okienko. Otoč Bežec rubovou stranou nahor a nadrež pozdĺž čiarkovaných čiar. Prehni na nadrezaných čiarach.

  \textbf{Krok 5 Okienko Bežca} Polopriehľadné okienko sa nachádza na hárku pauzovacieho papiera, označenom \textbf{PRIESVITNÉ ČASTI}.\footnote{Hárok obsahuje dve okienka Pevnej časti a dve okienka Bežca. Potrebuješ jedno z každého pre výrobu jedného pravítka.} Pretože niektoré obrysové čiary by ovplyvňovali funkciu Bežca, sú len naznačené v rohoch okienka číselným poradím rezov ( 1, 2, 3, 4 ). Najkôr urob nožom a pravítkom rovný rez medzi čiarkami označenými 1, potom zopakuj medzi čiarkami 2, atď., až kým nevyrežeš celé obdĺžnikové okienko. Nanes štetcom lepidlo na vyšrafované plochy vo vnútri tela Bežca (vyrobené v predchádzajucom kroku) a opatrne na ne pritlač priesvitné okienko. Vlož Bežec medzi dva hárky starého papiera, zaťaž knihami a nechaj jeden deň schnúť.

  \textbf{Krok 6 Okienko Pevnej časti} Vyrež zvyšný diel na hárku označenom \textbf{PRIESVITNÉ ČASTI}.\footnotemark[\value{footnote}] Rež nožom a pravítkom pozdĺž plných čiar. Na vnútornej strane tela Pevnej časti sú šrafovaním vyznačené úzke plochy pre lepidlo. Nanes štecom lepidlo na jeden z vyšrafovaných prúžkov a umiestni naň priesvitný diel. Uhlaď prstami, umiestni medzi dva hárky starého papiera a zaťaž knihami. Nechaj jeden deň schnúť.

  \textbf{Krok 7 Dokonči Pevnú časť} Nanes štetcom lepidlo na druhý šrafovaný pruh na vnútornej strane tela Pevnej časti. Veľmi opatrne a presne uzavri telo Pevnej časti tak, aby boli predná a zadná strana rovné. Zasuň Posuvnú časť do Pevnej časti a zaťaž knihami. Nechaj jeden deň schnúť.
Môže sa stať, že okienko Pevnej časti bude presahovať na jednej alebo druhej strane, alebo na oboch. Použi nôž alebo nožnice a odstráň presah.

  \textbf{Krok 8 Dokonči Bežec} Bežec sa posúva po vonkajšom povrchu Pevnej časti. Navleč ho na dokončenú Pevnú časť a nanes štetcom lepidlo na šrafovanú plôšku na zadnej strane Bežca. Spoj zadnú stranu Bežca, uhlaď prstami a zaťaž dokončené Logaritmické pravítko knihami. Nechaj jeden deň schnúť.

  \textbf{Krok 9 Skladací návod} V tomto zošite je ďalší návod, nazvaný \textbf{Ako používať Logaritmické pravítko}, na ktorom sú štyri čiarkované sivé čiary.\footnote{Zošit obsahuje dve kópie spomínaného návodu, jednu pre každé pravítko.} Odstrihni alebo odrež ľavý a pravý okraj návodu podľa vonkajších čiarkovaných čiar. Prehni pozdĺž zostávajúcich dvoch čiar do tvaru leporela. Takto má návod ideálny tvar a rozmer na to, aby sa dal vsunúť do vnútra Posuvnej časti pravítka. Môžeš ho tam dokonca za jednu stranu prilepiť.

  \textbf{Krok 10 Nauč sa používať Logaritmické pravítko} Prečítaj si inštrukcie v návode \textbf{Ako používať Logaritmické pravítko} a nauč sa používať svoj analógový počítač.

  }
  \end{multicols*}
  
\end{document}
