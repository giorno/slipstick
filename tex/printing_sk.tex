% common packages, fonts, styles and commands
% unlocalised content
\documentclass[a4paper,9pt]{article}
\usepackage{fontspec}
\usepackage[a4paper,lmargin=2cm,rmargin=2cm,bottom=2cm,top=2cm]{geometry}
\usepackage{multicol}
\usepackage{titling}
\usepackage{fancyhdr}
\usepackage{paralist}
\usepackage{gensymb}

\setmainfont{Roboto Slab Light}

% section without title
\newcommand{\nosection}[1]{%
  \refstepcounter{section}%
  \addcontentsline{toc}{section}{\protect\numberline{\thesection}#1}%
  \markright{#1}}

% title formatting
\newfontfamily\headingfont[]{Oswald}
\newfontfamily\constfont[]{Roboto}
\newcommand{\constt}[1]{{\constfont\selectfont{\textit{#1}}}}
\renewcommand{\maketitlehooka}{\headingfont}
\setlength{\droptitle}{-5em}

% paragraph spacing
\setlength{\parindent}{0em}
\setlength{\parskip}{0.5em}


\newcommand{\makefulltitle}{Ako vytlačiť Logaritmické pravítko }
% custom page number
\fancypagestyle{plain}{% title page
  \renewcommand{\headrulewidth}{0pt}%
  \fancyhf{}%
  \fancyfoot[L]{\headingfont{http://wheel.creat.io/sr}}%
  \fancyfoot[R]{\headingfont{\makefulltitle \textbf{\thepage}}}%
  \renewcommand{\headrulewidth}{0pt}
}
\pagestyle{fancy}% other pages
\fancyhf{}
\lfoot{\headingfont{http://wheel.creat.io/sr}}
\rfoot{\headingfont{\makefulltitle \textbf{\thepage}}}
\renewcommand{\headrulewidth}{0pt}


% common part of how-to perex
\newcommand{\makeperex}{Logaritmické pravítko je analógový počítač - zariadenie, ktoré dokáže počítať väčšinu základných a niektoré pokročilé matematické operácie. Nie je síce také presné a nepoužíva sa tak jednoducho ako kalkulačka, ale elegantne demonštruje matematické zákony. Je to historický artefakt, ktorý bol ešte pred päťdesiatimi rokmi bežne používaný v technických disciplínach, inžinieri a astronauti ním počas vesmírnych misií dokonca počítali trajektórie a letové parametre. }
% custom page number
\fancypagestyle{plain}{% title page
  \renewcommand{\headrulewidth}{0pt}%
  \fancyhf{}%
  \fancyfoot[L]{\headingfont{http://wheel.creat.io/sr}}%
  \fancyfoot[R]{\headingfont{\makefulltitle \textbf{\thepage}}}%
  \renewcommand{\headrulewidth}{0pt}
}
\pagestyle{fancy}% other pages
\fancyhf{}
\lfoot{\headingfont{http://wheel.creat.io/sr}}
\rfoot{\headingfont{\makefulltitle \textbf{\thepage}}}
\renewcommand{\headrulewidth}{0pt}





\title{\fontsize{60}{60}\selectfont LOGARITMICKÉ PRAVÍTKO}
\preauthor{}\postauthor{}\author{}
\predate{}\postdate{}\date{}
\begin{document}

  \begin{center}
    \headingfont\fontsize{20}{20}\selectfont AKO VYTLAČIŤ
  \end{center}

  {\let\newpage\relax\maketitle}% no new line before title
  \nosection{}
  \large\textbf{\makeperex Tento extrémne stručný návod opisuje ako vytlačiť obsah tohoto dokumentu na papierové hárky pre výrobu papierového Logaritmického pravítka označeného ako \modelname.}

  \begin{multicols*}{3}
  \normalsize{

  Pri dostatočnom množstve papiera si môžeš vytlačiť a vyrobiť nekonečné množstvo Logaritmických pravítok. Tento dokument je poskytnutý pod licenciou Creative Commons Attribution-ShareAlike 4.0 International License, čo znamená, že ho môžeš šíriť a zdieľať, tlačený alebo v elektronickej forme, s kýmkoľvek. Sprav to pre nás!

  Na vytlačenie jedného setu hárkov pre výrobu jedného papierového Logaritmického pravítka potrebuješ:
    \begin{enumerate}
      \setlength{\parskip}{0pt}
      \setlength{\parsep}{0pt}
      \item laserovú tlačiareň (atramentová bude stačiť, ale laserová je ideálna)
      \item 2 A4 hárky 160-210 g/m (tvrdého) papiera (vhodného pre použitý typ tlačiarne)
      \item 1 A4 hárok pauzovacieho papiera (tiež vhodného pre tlačiareň)
    \end{enumerate}

  Ak sa rozhodneš tlačiť aj návody (Ako vyrobiť a Ako používať), môžeš na to použiť akýkoľvek druh papiera. Ak tvoj ovládač/tlačová aplikácia podporuje mierku, nastav ju na 100\%. Vypni všetky možnosti automatického centrovania.

  \textbf{Tvrdé hárky}
Celkom 4 strany sa tlačia na obe strany oboch hárkov tvrdého papiera. Lícne strany týchto hárkov sú označené \textbf{PEVNÁ ČASŤ + BEŽEC} a \textbf{POSUVNÁ ČASŤ}. Použi obojstrannú tlač.

  \textbf{Pauzovací papier}
Len jedna strana sa tlačí na jednu stranu hárku pauzovacieho papiera. Je označená \textbf{PRIESVITNÉ ČASTI}.

  \textbf{Čo ďalej}
Po vytlačení hárkov, pokračuj podľa inštrukcií v návode \textbf{Ako vyrobiť Logaritmické pravítko}.

  }
  \end{multicols*}
  
\end{document}
