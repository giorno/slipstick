\documentclass[a4paper,9pt]{article}
\usepackage{fontspec}
\usepackage[margin=1.5cm,bottom=2cm,top=2cm]{geometry}
\usepackage{multicol}
\usepackage{titling}
\usepackage{fancyhdr}

\setmainfont{Roboto Slab Light}

% section without title
\newcommand{\nosection}[1]{%
  \refstepcounter{section}%
  \addcontentsline{toc}{section}{\protect\numberline{\thesection}#1}%
  \markright{#1}}

% title formatting
\newfontfamily\headingfont[]{Oswald}
\renewcommand{\maketitlehooka}{\headingfont}
\setlength{\droptitle}{-5em}

% custom page number
\fancypagestyle{plain}{% title page
  \renewcommand{\headrulewidth}{0pt}%
  \fancyhf{}%
  \fancyfoot[L]{\headingfont{https://github.com/giorno/slipstick}}%
  \fancyfoot[R]{\headingfont{How To Use The Slide Rule \textbf{\thepage}}}%
  \renewcommand{\headrulewidth}{0pt}
}
\pagestyle{fancy}% other pages
\fancyhf{}
\lfoot{\headingfont{https://github.com/giorno/slipstick}}
\rfoot{\headingfont{How To Use The Slide Rule \textbf{\thepage}}}
\renewcommand{\headrulewidth}{0pt}

% paragraph spacing
\setlength{\parindent}{0em}
\setlength{\parskip}{0.5em}


\title{\fontsize{60}{60}\selectfont THE SLIDE RULE}
\preauthor{}\postauthor{}\author{}
\predate{}\postdate{}\date{}
\begin{document}

  \begin{center}
    \headingfont\fontsize{20}{20}\selectfont HOW TO USE
  \end{center}

  {\let\newpage\relax\maketitle}% no new line before title
  \nosection{}
  \large\textbf{Slide Rule is an analog computer, a device that you can use to perform all basic and some advanced mathematical operations. It is not as precise as pocket calculator, but does not require electrical power. It was commonly used even 50 years ago, the time when they started being replaced by pocket calculators. Slide rules were one of many instruments used by engineers and astronauts in space missions for flight parameters calculation. This guide describes how to use a paper made Slide Rule designated as Model A.}

  \begin{multicols*}{3}
  %\normalsize{Before you start, you will need to get your hands on the following:
  %  \begin{enumerate}
  %    \setlength{\parskip}{0pt}
  %    \setlength{\parsep}{0pt}
  %    \item scissors
  %    \item sharp trimming knife (or use scissors instead) to cut out the parts, be careful when using it
  %    \item ruler (preferably metal one) for cutting straight lines
  %    \item PVA glue
  %    \item narrow brush for gluing
  %    \item one sheet of 300 g/m card paper to strengthen the back of the Slide Rule
  %    \item heavy books to press glued surfaces
  %  \end{enumerate}

  %This booklet contains two pages printed on thick paper and one on tracing paper (semitransparent one). From them you will cut out, fold and glue together \textbf{parts} of the Slide Rule.

  The Slide Rule consists of three components:
    \begin{enumerate}
      \setlength{\parskip}{0pt}
      \setlength{\parsep}{0pt}
      \item stock
      \item slide rule
      \item cursor
    \end{enumerate}

  On the Stock face and both sides of the Slide Rule there are logarithmic scales marked by capital letters. On the back of the Stock there are tables of mathematical and physical constants, and other useful graphics, but they are not involved in calculations. The Slide Rule contains, besides the logarithmic scales, conversion ratios between imperial and metric (SI) units.

  The Slide Rule slides inside the Stock and the Cursor slides on the top of the Stock. By sliding these parts, aligning the scales to appropriate positions and reading them, mathematical operations are performed.

  \textbf{Simple Multiplication} Scales labeled C, D, A, B are used for multiplication.

  \textbf{Wrap-around Multiplication} Scales labeled C, D, A, B are used for multiplication.

  \end{multicols*}
  
\end{document}
