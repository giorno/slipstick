% common packages, fonts, styles and commands
% unlocalised content
\documentclass[a4paper,9pt]{article}
\usepackage{fontspec}
\usepackage[a4paper,lmargin=2cm,rmargin=2cm,bottom=2cm,top=2cm]{geometry}
\usepackage{multicol}
\usepackage{titling}
\usepackage{fancyhdr}
\usepackage{paralist}
\usepackage{gensymb}

\setmainfont{Roboto Slab Light}

% section without title
\newcommand{\nosection}[1]{%
  \refstepcounter{section}%
  \addcontentsline{toc}{section}{\protect\numberline{\thesection}#1}%
  \markright{#1}}

% title formatting
\newfontfamily\headingfont[]{Oswald}
\newfontfamily\constfont[]{Roboto}
\newcommand{\constt}[1]{{\constfont\selectfont{\textit{#1}}}}
\renewcommand{\maketitlehooka}{\headingfont}
\setlength{\droptitle}{-5em}

% paragraph spacing
\setlength{\parindent}{0em}
\setlength{\parskip}{0.5em}


\usepackage{polyglossia}
\setdefaultlanguage{slovak}
\usepackage{background}
\usetikzlibrary{calc}
\usepackage{fixltx2e}

% lines for cutting and bending for folded book like manual
\backgroundsetup{
color=black,
scale=1,
opacity=1,
angle=0,
contents={\tikz\draw[line width=0.1mm, gray]
    (0mm, 0mm) -- (0mm, 297mm)
    (180mm, 0mm) -- (180mm, 297mm)
    [dashed]
    (60mm, 0mm) -- (60mm, 297mm)
    (120mm, 0mm) -- (120mm, 297mm);
  }
}

\newcommand{\makefulltitle}{Ako používať Logaritmické pravítko }
% custom page number
\fancypagestyle{plain}{% title page
  \renewcommand{\headrulewidth}{0pt}%
  \fancyhf{}%
  \fancyfoot[L]{\headingfont{http://wheel.creat.io/sr}}%
  \fancyfoot[R]{\headingfont{\makefulltitle \textbf{\thepage}}}%
  \renewcommand{\headrulewidth}{0pt}
}
\pagestyle{fancy}% other pages
\fancyhf{}
\lfoot{\headingfont{http://wheel.creat.io/sr}}
\rfoot{\headingfont{\makefulltitle \textbf{\thepage}}}
\renewcommand{\headrulewidth}{0pt}


% common part of how-to perex
\newcommand{\makeperex}{Logaritmické pravítko je analógový počítač - zariadenie, ktoré dokáže počítať väčšinu základných a niektoré pokročilé matematické operácie. Nie je síce také presné a nepoužíva sa tak jednoducho ako kalkulačka, ale elegantne demonštruje matematické zákony. Je to historický artefakt, ktorý bol ešte pred päťdesiatimi rokmi bežne používaný v technických disciplínach, inžinieri a astronauti ním počas vesmírnych misií dokonca počítali trajektórie a letové parametre. }
% custom page number
\fancypagestyle{plain}{% title page
  \renewcommand{\headrulewidth}{0pt}%
  \fancyhf{}%
  \fancyfoot[L]{\headingfont{http://wheel.creat.io/sr}}%
  \fancyfoot[R]{\headingfont{\makefulltitle \textbf{\thepage}}}%
  \renewcommand{\headrulewidth}{0pt}
}
\pagestyle{fancy}% other pages
\fancyhf{}
\lfoot{\headingfont{http://wheel.creat.io/sr}}
\rfoot{\headingfont{\makefulltitle \textbf{\thepage}}}
\renewcommand{\headrulewidth}{0pt}





\title{\fontsize{60}{60}\selectfont LOGARITMICKÉ PRAVÍTKO}
\preauthor{}\postauthor{}\author{}
\predate{}\postdate{}\date{}
\begin{document}

  \begin{center}
    \headingfont\fontsize{20}{20}\selectfont AKO POUŽÍVAŤ
  \end{center}

  {\let\newpage\relax\maketitle}% no new line before title
  \nosection{}
  \large\textbf{\makeperex Tento návod popisuje ako používať papierové Logaritmické pravítko \modelname.}

  \begin{multicols*}{3}
  \normalsize{
  Logaritmické pravítko sa skladá z:
    \begin{inparaenum}[a\upshape)]
      \item Pevnej časti;
      \item Posuvnej časti; a
      \item Bežca.
    \end{inparaenum}

  Na prednej strane Pevnej časti a oboch stranách Posuvnej časti sú zobrazené logaritmické stupnice, označené veľkými písmenami. Stupnice na Pevnej časti:
  \begin{inparaenum}[a\upshape)]
    \item \textbf{L, P, K, A} na hornej polovici; a
    \item \textbf{D, S, T, ST} na spodnej polovici.
  \end{inparaenum}
  Stupnice na Posuvnej časti:
  \begin{inparaenum}[a\upshape)]
    \item \textbf{B, CI, C} na jednej strane; a
    \item \textbf{LL1, LL2, LL3} na opačnej.
  \end{inparaenum}

  Čiary na stupnici sa nazývajú rysky alebo \textbf{indexy}. Číselné značky na stupnici označujú významné indexy.

  Na zadnej strane Pevnej časti sú tabuľky matematických a fyzikálnych konštánt a ďalšia užitočná grafika\footnote{Metrické a palcové stupnice na zadnej strane Pevnej časti a prednej strane Bežca nemusia merať presne, pretože teplota a vlhkosť spôsobené lepením menia mechanické vlastnosti a rozmery papiera.}, ale tieto sa nepoužívajú v matematických operáciách. Posuvná časť obsahuje prevodné mierky pre imperiálne a metrické (SI) jednotky a na vnútorných stranách prevodné stupnice medzi číselnými sústavami a mierami uhlov.

  Posuvná časť sa pohybuje vo vnútri Pevnej časti a Bežec po jej povrchu. Matematické operácie sa vykonávajú vzájomným posúvaním častí, zarovnaním indexov stupníc a odčítaním hodnôt. Niektoré operácie vyžadujú drobnú konverziu desatinných miest.

  \textbf{Jednoduché násobenie}
\constt{1.2{\char"00D7}2.3:}
Posuň index 1 stupnice C na pozíciu indexu 1.2 stupnice D.
Posuň Bežec na index 2.3 stupnice D.
Hodnota, ktorú Bežec ukazuje na stupnici C je výsledok: \constt{2.76}.
\footnotesize Použi stupnice A a B pre násobenie a delenie väčších čísel.
\normalsize

  \textbf{Násobenie s prechodom}
\constt{2.4{\char"00D7}4.6}
Posuň index 10 stupnice C na pozíciu indexu 2.4 stupnice D.
Posuň Bežec na index 4.6 stupnice C.
Hodnota, ktorú Bežec ukazuje na stupnici D je 1.105.
Z pamäti vypočítaj, že súčin približných hodnôt 2.5{\char"00D7}5 je 12.5.
Výsledok je teda väčší ako 10.
Úpravou desatinného miesta získaš výsledok: \constt{11.05}. 

  \textbf{Jednoduché delenie}
\constt{4.6/7.7:}
Posuň Bežec na index 4.6 stupnice D.
Posuň index 7.7 stupnice C na Bežec.
Posuň Bežec buď na index 1 alebo index 10 stupnice C, podľa toho, ktorý je bližšie. V tomto prípade index 10.
Bežec je teraz na indexe 5.97 na stupnici D. Správna odpoveď je približne 4 / 8 = 0.5. Úpravou desatinného miesta získaš výsledok: \constt{0.597}.

  \textbf{Recipročná hodnota}
\constt{1/7.6:}
\footnotesize Stupnica CI je inverzná stupnica, hodnoty indexov stúpajú zprava doľava. \normalsize
Posuň Bežec na index 7.8 stupnice CI.
Bežec je teraz na indexe 1.132 na stupnici C.
Správna odpoveď je približne 1 / 10 = 0.1. Úpravou desatinného miesta získaš výsledok: \constt{0.1132}.

  \textbf{Druhá mocnina}
\constt{4.2\textsuperscript{2}:}
Posuň Bežec na index 4.2 stupnice C.
Bežec ukazuje výsledok \constt{17.6} na stupnici B.

  \textbf{Druhá odmocnina}
\footnotesize Pre počítanie druhej odmociny čísel s nepárnym počnom číslic použi rozsah [1, 10] stupnice B. Pre druhú mocninu čísel s párnym počtom číslic použi rozsah [10, 100] stupnice B. \normalsize
\constt{4400\textsuperscript{0.5}:}
Posuň Bežec na index 44 stupnice B (4400 má párny počet číslic).
Bežec je na indexe 6.65 na strupnici C. Správna odpoveď je približne 70, keďźe 70\textsuperscript{2} = 4900. Úpravout desatinného miesta získaš výsledok: \constt{66.5}.

  \textbf{Tretia mocnina}
\constt{4.8\textsuperscript{3}:}
Posuň Bežec na index 4.8 stupnice D.
Bežec ukazuje výsledok \constt{110} na stupnici K.

  \textbf{Tretia odmocnina}
\footnotesize Rozsah [1, 10] stupnice K sa používa pre tretie odmocniny čísel s jednou číslicou, rozsah [10, 100] pre čísla s dvoma číslicami, [100, 1000] pre trojmiestne čísla. Čísla so štyrmi číslicami opäť prvý rozsah, päť číslic druhý rozsah, atď. \normalsize
\constt{4400\textsuperscript{0.33}:}
Posuň Bežec na index 4.4 stupnice K (4400 má 4 číslice).
Bežec je na indexe 1.655 na stupnici D.
Správna odpoveď je medzi 10 a 20 pretože 10\textsuperscript{3} = 1000 a 20\textsuperscript{3} = 8000. Úpravou desatinného miesta získaš výsledok: \constt{16.55}.

  \textbf{Desiata mocnina}
\constt{1.36\textsuperscript{10}:}
\footnotesize Mocnenie na mocniny čísla 10 a všeobecné mocniny (vrátane odmocnín) sa počítajú na stupniciach LL, ktoré sú umiestnené na opačnej strane Posuvnej časti. \normalsize
Posuň Bežec na index 1.36 stupnice LL2.
Bežec sa nachádza na indexe \constt{21.5} na stupnici LL3, čo je výsledok operácie.

  \textbf{Stá mocnina}
\constt{1.02\textsuperscript{100}:}
Posuň Bežec na index 1.02 stupnice LL1.
Bežec sa nachádza na indexe \constt{7.25} na stupnici LL3, čo je výsledok operácie.

  \textbf{Desiata odmocnina}
\constt{5\textsuperscript{0.1}:}
Posuň Bežec na index 5 stupnice LL3.
Bežec sa nachádza na indexe \constt{1.175} na stupnici LL2, čo je výsledok operácie.

  \textbf{Stá odmocnina}
\constt{5\textsuperscript{0.01}:}
Posuň Bežec na index 5 stupnice LL3.
Bežec sa nachádza na indexe \constt{1.022} na stupnici LL1, čo je výsledok operácie.

  \textbf{Všeobecná mocnina}
\constt{9.1\textsuperscript{2.3}:}
Posuň index 9.1 stupnice LL3 na pozíciu indexu 1 stupnice D.
Posuň Bežec na index 2.3 stupnice D.
Pozícia Bežca na stupnici LL3 je výsledok: \constt{160.6}. Bežec je v skutočnosti v blízkosti 161, pretože stupnica stráca presnosť so stúpajúcou hodnotou. 

\constt{230\textsuperscript{0.45}:}
Posuň index 230 stupnice LL3 na pozíciu indexu 10 stupnice C.
Posuň Bežec na index 4.5 stupnice C.
Bežec sa nachádza na indexe výsledku na stupnici LL3: \constt{11.6}.

\constt{0.78\textsuperscript{3.4}:}
Posuň index 3.4 stupnice LL3 na pozíciu indexu 7.8 stupnice C.
Posuň Bežec na index 10 stupnice C.
Bežec sa nachádza na indexe výsledku na stupnici LL3: \constt{4.3}.

\constt{1.9\textsuperscript{2.5}:}
Posuň index 1.9 stupnice LL2 na pozíciu indexu 10 stupnice C.
Posuň Bežec na index 2.5 stupnice C.
Bežec sa nachádza na indexe výsledku na stupnici LL3: \constt{4.97}.

\constt{0.99\textsuperscript{560}:}
Posuň index 9.9 stupnice LL3 na pozíciu indexu 10 stupnice C.
Posuň Bežec na index 5.6 stupnice C.
Bežec sa nachádza na indexe 3.6 na stupnici LL3.
Pretože si použil/použila jednu stotinu exponentu, úpravou hodnoty o dve desatinné miesta získaš výsledok: \constt{0.036}.

  \textbf{Sínus}
\footnotesize Použi stupnicu S pre výpočet sínusu uhlov medzi 5\textdegree a 90\textdegree, a stupnicu ST pre uhly medzi 30' a 6\textdegree. \normalsize
\constt{sin(33\textdegree):}
Posuň Bežec na 33 stupnice S.
Bežec sa nachádza na indexe 5.45 na stupnici C.
Správna odpoveď je medzi 0.1 a 1. Úpravou desatinného miesta získaš výsledok: \constt{0.545}.

  \textbf{Kosínus}
\constt{cos(33\textdegree):}
Posuň Bežec na index 33 stupnici S.
Bežec sa nachádza na indexe výsledku \constt{0.83} na stupnici P (inverzná stupnica).

  \textbf{Tangens}
\footnotesize Použi stupnicu T pre výpočet tangensu uhlov medzi 5\textdegree a 45\textdegree, a stupnicu ST pre uhly medzi 30' a 6\textdegree. \normalsize
\constt{tan(33\textdegree):}
Posuň Bežec na index 33 stupnice S.
Bežec sa nachádza na indexe 6.5 na stupnici C.
Správny výsledok je medzi 0.1 a 1. Úpravou desatinného miesta získaš výsledok: \constt{0.65}.

  \textbf{Dekadický logaritmus}
\constt{log\textsubscript{10}2.6:}
Posuň Bežec na index 2.6 stupnice C.
Bežec sa nachádza na indexe 4.14 na stupnici L.
Úpravou hodnoty o desatinné miesto získaš výsledok: \constt{0.414}.
  }
  \end{multicols*}
  
\end{document}
