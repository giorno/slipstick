% common packages, fonts, styles and commands
% unlocalised content
\documentclass[a4paper,9pt]{article}
\usepackage{fontspec}
\usepackage[a4paper,lmargin=2cm,rmargin=2cm,bottom=2cm,top=2cm]{geometry}
\usepackage{multicol}
\usepackage{titling}
\usepackage{fancyhdr}
\usepackage{paralist}
\usepackage{gensymb}

\setmainfont{Roboto Slab Light}

% section without title
\newcommand{\nosection}[1]{%
  \refstepcounter{section}%
  \addcontentsline{toc}{section}{\protect\numberline{\thesection}#1}%
  \markright{#1}}

% title formatting
\newfontfamily\headingfont[]{Oswald}
\newfontfamily\constfont[]{Roboto}
\newcommand{\constt}[1]{{\constfont\selectfont{\textit{#1}}}}
\renewcommand{\maketitlehooka}{\headingfont}
\setlength{\droptitle}{-5em}

% paragraph spacing
\setlength{\parindent}{0em}
\setlength{\parskip}{0.5em}


\newcommand{\makefulltitle}{How To Print The Slide Rule }
% custom page number
\fancypagestyle{plain}{% title page
  \renewcommand{\headrulewidth}{0pt}%
  \fancyhf{}%
  \fancyfoot[L]{\headingfont{http://wheel.creat.io/sr}}%
  \fancyfoot[R]{\headingfont{\makefulltitle \textbf{\thepage}}}%
  \renewcommand{\headrulewidth}{0pt}
}
\pagestyle{fancy}% other pages
\fancyhf{}
\lfoot{\headingfont{http://wheel.creat.io/sr}}
\rfoot{\headingfont{\makefulltitle \textbf{\thepage}}}
\renewcommand{\headrulewidth}{0pt}


% common part of how-to perex
\newcommand{\makeperex}{Slide Rule is an analog computer, a device that you can use to perform most of the basic and some advanced mathematical operations. It is not as precise and easy to use as pocket calculator, but elegantly demonstrates the laws of mathematics. It is a historical artifact that was still commonly used 50 years ago in many fields of science and technology, engineers and astronauts in space missions used it for flight parameters and trajectory calculations. }
% custom page number
\fancypagestyle{plain}{% title page
  \renewcommand{\headrulewidth}{0pt}%
  \fancyhf{}%
  \fancyfoot[L]{\headingfont{http://wheel.creat.io/sr}}%
  \fancyfoot[R]{\headingfont{\makefulltitle \textbf{\thepage}}}%
  \renewcommand{\headrulewidth}{0pt}
}
\pagestyle{fancy}% other pages
\fancyhf{}
\lfoot{\headingfont{http://wheel.creat.io/sr}}
\rfoot{\headingfont{\makefulltitle \textbf{\thepage}}}
\renewcommand{\headrulewidth}{0pt}





\title{\fontsize{60}{60}\selectfont THE SLIDE RULE}
\preauthor{}\postauthor{}\author{}
\predate{}\postdate{}\date{}
\begin{document}

  \begin{center}
    \headingfont\fontsize{32}{32}\selectfont HOW TO PRINT
  \end{center}

  {\let\newpage\relax\maketitle}% no new line before title
  \nosection{}
  \large\textbf{\makeperex This extremely brief guide describes how to print the content of this document to prepare the paper sheets for making two paper made Slide Rules \modelname.}

  \begin{multicols*}{3}
  \normalsize{

  Giving sufficient amount of paper you can print and make any number of Slide Rules.

This document is Licensed under Creative Commons Attribution-ShareAlike 4.0 International License, which means that you can share it, either in printed form or electronically, with anyone. We encourage you to do so!

  For a set of printouts to make two Slide Rules you will need the following:
    \begin{enumerate}
      \setlength{\parskip}{0pt}
      \setlength{\parsep}{0pt}
      \item a laser printer (an inkjet printer can be used instead, that however means lower quality\footnote{You can always use services of a print shop.})
      \item four A4 sheets of 160-210 g{\char"00B7}m\textsuperscript{-2} (thick) paper (suitable for the type of the printer)
      \item one A4 sheet of tracing paper (also suitable for the type of the printer)\footnote{If the printer cannot print on tracing paper, you can transfer dimensions of the transparent parts manually, using pencil and ruler.}
    \end{enumerate}

Optionally you can also print the title page, and the how-to guides for making and using the Slide Rule on whatever paper type you choose. 

If your print manager or driver supports scaling, set the scale to 100\%. If possible, use borderless printing.

  \makesectiontitle{Thick Paper Printouts}

There are 8 pages (pages 1-8) that have to be printed onto both sides of four sheets of thick paper.

Front pages of those sheets are labelled \textbf{STOCK + CURSOR} and \textbf{SLIDE}.

Use duplex printing.

  \makesectiontitle{Tracing Paper Printout}

There is 1 page (page 9) that has to be printed on one side of tracing paper sheet.

  \makesectiontitle{Follow Up}

Once printed, use the printouts as described in \textbf{How To Make The Slide Rule}.

  }
  \end{multicols*}

\end{document}
