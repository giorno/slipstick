% common packages, fonts, styles and commands
% unlocalised content
\documentclass[a4paper,9pt]{article}
\usepackage{fontspec}
\usepackage[a4paper,lmargin=2cm,rmargin=2cm,bottom=2cm,top=2cm]{geometry}
\usepackage{multicol}
\usepackage{titling}
\usepackage{fancyhdr}
\usepackage{paralist}
\usepackage{gensymb}

\setmainfont{Roboto Slab Light}

% section without title
\newcommand{\nosection}[1]{%
  \refstepcounter{section}%
  \addcontentsline{toc}{section}{\protect\numberline{\thesection}#1}%
  \markright{#1}}

% title formatting
\newfontfamily\headingfont[]{Oswald}
\newfontfamily\constfont[]{Roboto}
\newcommand{\constt}[1]{{\constfont\selectfont{\textit{#1}}}}
\renewcommand{\maketitlehooka}{\headingfont}
\setlength{\droptitle}{-5em}

% paragraph spacing
\setlength{\parindent}{0em}
\setlength{\parskip}{0.5em}


\newcommand{\makefulltitle}{How To Make The Slide Rule }
% custom page number
\fancypagestyle{plain}{% title page
  \renewcommand{\headrulewidth}{0pt}%
  \fancyhf{}%
  \fancyfoot[L]{\headingfont{http://wheel.creat.io/sr}}%
  \fancyfoot[R]{\headingfont{\makefulltitle \textbf{\thepage}}}%
  \renewcommand{\headrulewidth}{0pt}
}
\pagestyle{fancy}% other pages
\fancyhf{}
\lfoot{\headingfont{http://wheel.creat.io/sr}}
\rfoot{\headingfont{\makefulltitle \textbf{\thepage}}}
\renewcommand{\headrulewidth}{0pt}


% common part of how-to perex
\newcommand{\makeperex}{Slide Rule is an analog computer, a device that you can use to perform most of the basic and some advanced mathematical operations. It is not as precise and easy to use as pocket calculator, but elegantly demonstrates the laws of mathematics. It is a historical artifact that was still commonly used 50 years ago in many fields of science and technology, engineers and astronauts in space missions used it for flight parameters and trajectory calculations. }
% custom page number
\fancypagestyle{plain}{% title page
  \renewcommand{\headrulewidth}{0pt}%
  \fancyhf{}%
  \fancyfoot[L]{\headingfont{http://wheel.creat.io/sr}}%
  \fancyfoot[R]{\headingfont{\makefulltitle \textbf{\thepage}}}%
  \renewcommand{\headrulewidth}{0pt}
}
\pagestyle{fancy}% other pages
\fancyhf{}
\lfoot{\headingfont{http://wheel.creat.io/sr}}
\rfoot{\headingfont{\makefulltitle \textbf{\thepage}}}
\renewcommand{\headrulewidth}{0pt}





\title{\fontsize{60}{60}\selectfont THE SLIDE RULE}
\preauthor{}\postauthor{}\author{}
\predate{}\postdate{}\date{}
\begin{document}

  \begin{center}
    \headingfont\fontsize{20}{20}\selectfont HOW TO MAKE
  \end{center}

  {\let\newpage\relax\maketitle}% no new line before title
  \nosection{}
  \large\textbf{\makeperex This guide describes how to make a paper made Slide Rule designated as Model A.}

  \begin{multicols*}{3}
  \normalsize{Before you start, you will need to get your hands on the following:
    \begin{enumerate}
      \setlength{\parskip}{0pt}
      \setlength{\parsep}{0pt}
      \item scissors
      \item sharp hobby knife (or use scissors instead) to cut out the parts and ruler (preferably metal one) for cutting straight lines
      \item PVA glue
      \item narrow brush for gluing
      \item one A4 sheet of 300 g/m card paper to strengthen the back of the Slide Rule
      \item at least 4 A4 sheets of scrap paper or old newspaper
      \item heavy books to press glued surfaces
    \end{enumerate}

  \textbf{Ground Rules} to save you some injuries and unwanted results:
    \begin{inparaenum}[\itshape a\upshape)]
      \item be careful with the knife and the scissors;
      \item use the knife and the ruler to achieve perfectly straight cuts;
      \item be patient and pay attention to detail;
      \item wash and dry your hands before you begin;
      \item remove any excess glue;
      \item do not mix the glue with water; and
      \item do not delay pressing glued parts together after the glue is applied.
    \end{inparaenum}

  This booklet contains two sheets printed on thick paper, labeled \textbf{STOCK + CURSOR} and \textbf{SLIDE}, and one on tracing paper, labeled \textbf{TRANSPARENT}. From them you will cut out, pre-bend, fold and glue together \textbf{the components} of the Slide Rule:
    \begin{inparaenum}[\itshape a\upshape)]
      \item Stock;
      \item Slide; and
      \item Cursor.
    \end{inparaenum}

  \textbf{Step 1 Begin the Stock part} Cut out the Stock part. It is printed on the paper sheet labeled \textbf{STOCK + CURSOR} and it is the biggest part on that sheet. Cut along the outer lines (large rectangle) printed on the face side of the sheet using the knife and the ruler (or simply use scissors). Turn the part overleaf and use slight pressure of the knife to score along four dashed lines. Fold the stock along those lines.

  \textbf{Step 2 Strengthen the back of the Stock} From the card paper cut out a strip with dimensions 62x277 mm. Put thin layer of the glue on the big cross-hatched area of the Stock part and place the cardboard strip onto it precisely. Smooth the paper down with your fingers. Put the glued parts between two sheets of scrap paper and weight them under books for a day.

  \textbf{Step 3 Make the Slide part} The Slide component is printed on paper sheet labeled \textbf{SLIDE}. Cut along the two continuous lines on the reverse of the sheet from one edge of the sheet to the other. Score the part along the dashed line with the knife. Fold the part along that line. The Slide part is finished now. One out of the three!

  \textbf{Step 4 Begin the Cursor part} Cut out the Cursor body from the sheet labeled \textbf{STOCK + CURSOR} using the knife and the ruler along the outer lines. Use the scissors to cut out the arc window. Turn the cursor overleaf and score the paper along the dotted lines. Fold the cursor.

  \textbf{Step 5 Cursor window} This is the semitransparent part of the Cursor. Cut it out of the sheet labeled \textbf{TRANSPARENT}. Because some of the lines would interfere with the function of the window, they are only hinted at the corners. Hints are labeled with numbers 1, 2, 3 and 4 to assist with cutting order. First cut along the ruler between lines marked 1, then again between lines marked 2, and so on, until a rectangular shape is cut out of the tracing paper sheet. Use the brush to apply glue on the hatched areas on the inside of the Cursor body (made in previous step) and carefully place the transparent rectangle on it. Put the Cursor between two sheets of scrap paper and weight it under books for a day.

  \textbf{Step 6 Stock window} This is the other part printed on the \textbf{TRANSPARENT} labeled sheet (tracing paper). Cut it out along the straight lines. Find narrow cross-hatched areas on the inside of the Stock body and apply glue on one of them using the brush. Place the transparent window on it and make sure that the edge of the transparent part is aligned with the Stock body. Smooth the paper down with your fingers, place them between sheets of scrap paper and weight under books for a day.

  \textbf{Step 7 Complete the Stock} Apply glue to the second narrow cross-hatched area on the inside of the Stock body. Very carefully complete the Stock by folding it so that it will connect with the window. Put the Slide part into the Stock and weight them under books for a day.
Trim them using either the knife or the scissors.
There may be some overlapping areas of the Stock window (tracing paper) protruding from the body. 

  \textbf{Step 8 Complete the Cursor} The Cursor slides on the outside of the Stock. Gird it around the now-finished Stock part. Apply glue on the cross-hatched area on the back of the Cursor and press together with the rest of the Cursor. Smooth the part down using your fingers and weight completed Slide Rule (Slide in Stock, Stock in Cursor) under books for a day.

  \textbf{Step 9 Folded Manual} There is a second instruction sheet, titled \textbf{How To Use The Slide Rule}, which has 4 dashed gray lines 
on each page. Cut out the margins along the outer lines. Bend the sheet along the remaining two dashed lines to form a folded-book-like manual (leporello). Now it has ideal size to put it inside the Slide and have it at hand. You may also fix it inside the Slide by glue.

  \textbf{Step 10 Learn how to Use the Slide Rule} Follow up with the instructions sheet \textbf{How To Use The Slide Rule} and lear how to use this brand new analog computer.

  }
  \end{multicols*}
  
\end{document}
