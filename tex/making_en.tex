% common packages, fonts, styles and commands
% unlocalised content
\documentclass[a4paper,9pt]{article}
\usepackage{fontspec}
\usepackage[a4paper,lmargin=1.7cm,rmargin=1.7cm,bottom=2cm,top=2cm]{geometry}
\usepackage{multicol}
\usepackage{titling}
\usepackage{fancyhdr}
\usepackage{paralist}
\usepackage{gensymb}
\setmainfont{Roboto Slab Light}

% model name
\newcommand{\modelname}{SR-M1A1}

% section without title
\newcommand{\nosection}[1]{%
  \refstepcounter{section}%
  \addcontentsline{toc}{section}{\protect\numberline{\thesection}#1}%
  \markright{#1}}

% title formatting
\newfontfamily\headingfont[]{Oswald}
\newfontfamily\constfont[]{Roboto}
\newcommand{\constt}[1]{{\constfont\selectfont{\textit{#1}}}}
\renewcommand{\maketitlehooka}{\headingfont}
\setlength{\droptitle}{-5em}

% paragraph spacing
\setlength{\parindent}{0em}
\setlength{\parskip}{0.5em}


\newcommand{\makefulltitle}{How To Make The Slide Rule }
% custom page number
\fancypagestyle{plain}{% title page
  \renewcommand{\headrulewidth}{0pt}%
  \fancyhf{}%
  \fancyfoot[L]{\headingfont{http://wheel.creat.io/sr}}%
  \fancyfoot[R]{\headingfont{\makefulltitle \textbf{\thepage}}}%
  \renewcommand{\headrulewidth}{0pt}
}
\pagestyle{fancy}% other pages
\fancyhf{}
\lfoot{\headingfont{http://wheel.creat.io/sr}}
\rfoot{\headingfont{\makefulltitle \textbf{\thepage}}}
\renewcommand{\headrulewidth}{0pt}


% common part of how-to perex
\newcommand{\makeperex}{Slide Rule is an analog computer, a device that you can use to perform all basic and some advanced mathematical operations. It is not as precise as pocket calculator, but does not require electrical power. It was commonly used even 50 years ago, the time when they started being replaced by pocket calculators. Slide rules were one of many instruments used by engineers and astronauts in space missions for flight parameters calculation. }
% custom page number
\fancypagestyle{plain}{% title page
  \renewcommand{\headrulewidth}{0pt}%
  \fancyhf{}%
  \fancyfoot[L]{\headingfont{http://wheel.creat.io/sr}}%
  \fancyfoot[R]{\headingfont{\makefulltitle \textbf{\thepage}}}%
  \renewcommand{\headrulewidth}{0pt}
}
\pagestyle{fancy}% other pages
\fancyhf{}
\lfoot{\headingfont{http://wheel.creat.io/sr}}
\rfoot{\headingfont{\makefulltitle \textbf{\thepage}}}
\renewcommand{\headrulewidth}{0pt}





\title{\fontsize{60}{60}\selectfont THE SLIDE RULE}
\preauthor{}\postauthor{}\author{}
\predate{}\postdate{}\date{}
\begin{document}

  \begin{center}
    \headingfont\fontsize{32}{32}\selectfont HOW TO MAKE
  \end{center}

  {\let\newpage\relax\maketitle}% no new line before title
  \nosection{}
  \large\textbf{\makeperex This guide describes how to make a paper made Slide Rule \modelname.}

  \begin{multicols*}{3}
\footnotesize  This booklet contains printouts, from which you will cut out, score, fold and glue together the components for two Slide Rules:
    \begin{inparaenum}[\itshape a\upshape)]
      \item two identical sheets labeled \textbf{STOCK + CURSOR}, providing Stock and Cursor bodies;
      \item two identical sheets labeled \textbf{SLIDE}, providing Slide bodies; and
      \item one sheet of tracing paper, labeled \textbf{TRANSPARENT}, providing transparent windows.
    \end{inparaenum}

  This document describes making of a single Slide Rule.

  \normalsize{To assemble the Slide Rule, you will need the following:
    \begin{enumerate}
      \setlength{\parskip}{0pt}
      \setlength{\parsep}{0pt}
      \item scissors
      \item sharp hobby knife (or use scissors instead) for cutting out the slide rule parts
      \item ruler (preferably a metal one) for cutting straight lines
      \item bone folder (or use knife instead) for scoring the paper
      \item thin dual-sided adhesive tape, 4 to 10 mm wide (it can be substituted by PVA glue at the expense the final instrument accuracy)
    \end{enumerate}

  To prevent injuries and unwanted results,
      be careful with the sharp tools,
      use the knife and the ruler to achieve perfectly straight cuts,
      be patient and pay attention to detail, and
      wash and dry your hands before you begin.

  \makesectiontitle{1 Begin the Stock part}

\footnotesize The Stock part is printed on the paper sheet labeled \textbf{STOCK + CURSOR} and it is the bigger of the two on that sheet.\normalsize

Cut along the outer lines (large rectangle) printed on the face side of the sheet using the knife and ruler.

Use the bone folder to score the paper between the hinted edges. There are two such edges to be scored, one between the scales \textbf{ST} and \textbf{cm}, and second between the scales \textbf{inches} and \textbf{L}.

Fold the Stock along those lines.

  \makesectiontitle{2 Make the Slide part}

\footnotesize The Slide component is printed on a paper sheet labeled \textbf{SLIDE}.\normalsize

 Cut along the two continuous lines on the reverse of the sheet from one edge of the sheet to the other.

 Score the part along the dashed line with the bone folder.

 Fold the part along that line. 

  \makesectiontitle{3 Stock window}

\footnotesize The Stock window part is printed on the \textbf{TRANSPARENT} labeled sheet (tracing paper). \normalsize

 Cut it out along the straight lines.

  \makesectiontitle{4 Complete the Stock part}

Find narrow cross-hatched areas on the inside of the Stock body and apply the dual sided adhesive tape on them.

Place the Slide (the side with the scales \textbf{B}, \textbf{CI} and \textbf{C} facing up) and the transparent window inside the Stock, the transparent part above the Slide, half close the Stock, so they are aligned with the edges.

Remove the cover film from the other side of the tape.

Carefully press upper half of the Stock face onto the transparent window.

Align the Slide so that index 1 on the scale B is aligned with index 1 on the scale A on the Stock.

Start pressing the bottom half of the Stock face onto the transparent window, starting from the left and making sure that index 1 on the scale C on the Slide is aligned with index 1 on the scale D on the Stock.

Continue the process until the whole length of the Stock is glued to the transparent window. Keep the parts flat while doing that.
 
  \makesectiontitle{5 Begin the Cursor part}

Cut out the Cursor body from a sheet labeled \textbf{STOCK + CURSOR} using the knife and the ruler along the outer lines.

Use the scissors to cut out the arc window.

Use the bone folder to score the paper between the hinted edges on the face of the Cursor. Fold it along those lines.

  \makesectiontitle{6 Cursor window}

\footnotesize This is the semitransparent part of the Cursor. Cut it out of the sheet labeled \textbf{TRANSPARENT}.\footnote{The sheet contains two Stock windows and two Cursor windows. You need one of each for single Slide Rule}\normalsize

Because some of the lines would interfere with the function of the window, they are only hinted at the corners. Hints are labeled with numbers 1, 2, 3 and 4 to assist with cutting order.

First cut along the ruler between lines marked 1, then again between lines marked 2, and so on, until a rectangular shape is cut out of the tracing paper sheet.

Apply short pieces of the double sided adhesive tape on the reverse of the Cursor near the circular edge and the opposite straight edge. Remove the cover film from the other side of the tape.

Carefully place the transparent window onto the reverse, making sure the edge is aligned with the cursor left and right edges. Press the parts together.

  \makesectiontitle{7 Complete the Cursor}

The Cursor slides on the outside of the Stock. Gird it around the Stock part, with the logo facing the front.

Apply double sided adhesive tape onto the cross-hatched area on the back of the Cursor, remove the covering film and press together with the back of the Cursor, leaving just enough space for the Cursor to slide on the Stock.

  \makesectiontitle{8 Folded Manual}

There is an instruction sheet, titled \textbf{How To Use The Slide Rule}, which has 4 dashed gray lines 
on each page.\footnote{There are two copies of that sheet in this booklet, one for each Slide Rule.} Cut out the margins along the outer lines. Bend the sheet along the remaining two dashed lines to form a folded-book-like manual (leporello). Now it has ideal size to put it inside the Slide and have it at hand. You may also fix it inside the Slide by glue.

  \makesectiontitle{9 Learn how to Use the Slide Rule}

Follow up with the instructions sheet \textbf{How To Use The Slide Rule} and lear how to use this brand new analog computer.

  }
  \end{multicols*}
  
\end{document}
