\documentclass[a4paper,10pt]{article}
\usepackage{fontspec}
\usepackage[margin=1.5cm,bottom=2cm,top=2cm]{geometry}
\usepackage{multicol}
\usepackage{titling}
\usepackage{fancyhdr}

\setmainfont{Roboto Slab Light}

% section without title
\newcommand{\nosection}[1]{%
  \refstepcounter{section}%
  \addcontentsline{toc}{section}{\protect\numberline{\thesection}#1}%
  \markright{#1}}

% title formatting
\newfontfamily\headingfont[]{Oswald}
\renewcommand{\maketitlehooka}{\headingfont}
\setlength{\droptitle}{-5em}

% custom page number
\fancypagestyle{plain}{% title page
  \renewcommand{\headrulewidth}{0pt}%
  \fancyhf{}%
  \fancyfoot[R]{\headingfont{How To Make The Slide Rule \textbf{\thepage}}}%
  \renewcommand{\headrulewidth}{0pt}
}
\pagestyle{fancy}% other pages
\fancyhf{}
\rfoot{\headingfont{How To Make The Slide Rule \textbf{\thepage}}}
\renewcommand{\headrulewidth}{0pt}

% paragraph spacing
\setlength{\parindent}{0em}
\setlength{\parskip}{1em}


\title{\fontsize{60}{60}\selectfont THE SLIDE RULE}
\preauthor{}\postauthor{}\author{}
\predate{}\postdate{}\date{}
\begin{document}

  \begin{center}
    \headingfont\fontsize{20}{20}\selectfont HOW TO MAKE
  \end{center}

  {\let\newpage\relax\maketitle}% no new line before title
  \nosection{}
  \large\textbf{Slide Rule is an analog computer, a device that you can use to perform all basic and some advanced mathematical operations. It is not as precise as pocket calculator, but does not require electrical power. It was commonly used even 50 years ago, the time when they started being replaced by pocket calculators. Slide rules were one of many instruments used by engineers and astronauts in space missions for flight parameters calculation. This guide describes how to make a paper made Slide Rule designated as Model A.}

  \begin{multicols*}{2}
  \normalsize{Before you start, you will need to get your hands on the following:
    \begin{enumerate}
      \setlength{\parskip}{0pt}
      \setlength{\parsep}{0pt}
      \item scissors
      \item sharp trimming knife (or use scissors instead) to cut out the parts, be careful when using it
      \item ruler (preferably metal one) for cutting straight lines
      \item PVA glue
      \item narrow brush for gluing
      \item one sheet of 300 g/m card paper to strengthen the back of the Slide Rule
      \item heavy books to press glued surfaces
    \end{enumerate}

  This booklet contains two pages printed on thick paper and one on tracing paper (semitransparent one). From them you will cut out, fold and glue together \textbf{parts} of the Slide Rule.

  The Slide Rule consists of three components:
    \begin{enumerate}
      \setlength{\parskip}{0pt}
      \setlength{\parsep}{0pt}
      \item stock
      \item slide rule
      \item cursor
    \end{enumerate}

  \textbf{Step 1 Begin the Stock part} Cut out the stock part. It is printed on paper sheet labeled \textbf{STOCK + CURSOR} and it is the biggest part on that sheet. Cut along the outer lines (large rectangle) printed on the face side of the sheet using the knife (or scissors) and the ruler. Turn the part over and use slight pressure (do not cut the paper through)of the knife to mark along dashed lines. There is four of them, marked by dashed line. Fold the stock along those lines after they were marked by knife (it will be easier that way).

  \textbf{Step 2 Strenghten the back of the Stock} From the card paper cut out a strip with dimensions 60x277 mm. Put thin layer of the glue on the big cross-hatched area of the stock part and place the cardboard strip onto it. Put the glued parts between two sheets of scrap paper, place it on a desk and put the heavy books onto it. Do not take the books away until the glue dries (overnight).

  \textbf{Step 3 Make the Slide part} Slide component is printed on paper sheet labeled \textbf{SLIPSTICK}. Cut along the two lines on the front of the sheet from one edge of the sheet to the other. On the reverse of the part there is a dashed lined, along which use slight pressure of knife to mark the folding. Fold the cut out part along that line. The Slide part is finished now. One out of the three!

  \textbf{Step 4 Begin the Cursor part} Printed on paper sheet labeled \textbf{STOCK + CURSOR}. Use the knife and the ruler to cut out the cursor body along the straight lines. Use the scissors to cut out the arc window. Turn the cursor overleaf and use slight pressure of knife to mark the folding lines. Fold the cursor.

  \textbf{Step 5 Cursor window} This is the semitransparent part of the Cursor. Cut it out of the sheet labeled \textbf{TRANSPARENT}. Use the brush to apply glue on the hatched areas on the inside of the cursor body and place the just cut out transparent rectangle on it. Put the cursor between two sheets of scrap paper and use heavy books to make pressure.

  \textbf{Step 6 Stock window} This is the other part printed on the \textbf{TRANSPARENT} labeled sheet (tracing paper). Cut it out along the straight lines...

  \textbf{Step 7 Complete the Stock}

  \textbf{Step 8 Complete the Cursor}

  }
  \end{multicols*}
  
\end{document}
